\subsection{Integer Linear Programming Model}

\subsubsection{Decision Variables}

For each vertex $\solutionE \in \vertices$ of the input graph $\graph = \tuple{\vertices, \edges}$, we define the binary variable $\varE \in \binary$ which indicates whether $\solutionE$ is in the solution $\solution$:

\begin{equation}
    \label{eq:decision-variables}
    \varE =  \begin{cases}
      1 &, \solutionE \in \solution \\
      0 &, \solutionE \notin \solution
   \end{cases}
\end{equation}

\subsubsection{Mathematical Model}

Below, \eqref{eq:ILP-objective} is the objective function: maximize the number of vertices in the solution.
\eqref{eq:ILP-capacity-constraint} is the Capacity-Constraint of \eqref{eq:capacity-constraint}.
\eqref{eq:ILP-order-constraint} is the Precedence-Constraint of \eqref{eq:precedence-constraint}: if a vertex $\solutionE$ is in the solution, then all vertices $\solutionEp$ ``above'' it must also be in the solution.

\begin{align}
    \label{eq:ILP-objective}
    \max\limits_{\solution \subseteq \vertices}
        & \Sum{\solutionE \in \solution}{}{\varE} \\
    s.t.
    \label{eq:ILP-capacity-constraint}
    & \Sum{\solutionE \in \solution}{}{\varE \weightE} \leqslant \maximumWeight \\
    \label{eq:ILP-order-constraint}
    & \var_{\solutionE} \leqslant \var_{\solutionEp} \quad \forall \solutionE \partialLower \solutionEp \\
    \label{eq:ILP-binary-constraint}
    & \var \in \varDomain
\end{align}
