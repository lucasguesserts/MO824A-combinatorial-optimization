\section{State of the Art}

In \cite{bib:grasp-and-tabu}, the authors proposes a memory based GRASP with restart and a simple tabu search algorithm is proposed to overcome the limitations of local optimality in order to find near optimal solutions. In that paper, numerical tests on benchmark instances demonstrate the effectiveness and efficiency of the proposed methodology which outperform the Mini-Swarm heuristic in terms of the success ratio, relative percentage deviation and computational time.

In \cite{bib:tabu-knapsack}, the authors reported the implementation of an efficient TS method based on the oscillation strategy and definition of a promising zone, a zone which englobes all factible solutions plus all infactible solutions bordering the infactible solutions, for solving the O-1 MKP which has been tested on standard test problems from \cite{bib:freville,bib:preprocessing-knapsack-1994} and \cite{bib:tabu-multidimensional-knapsack}. Optimal solutions were successfully obtained for all instances and the previously best known solutions were improved for five of the last seven instances. These numerical results were claimed to confirm the merit of tabu tunneling approaches to generate solutions of high quality for 0-1 multiknapsack problems. Moreover, these results (like those of \cite{bib:tabu-multidimensional-knapsack}) are claimed to establish that the oscillation strategy is efficient to balance the interaction between intensification and diversification strategies of TS.

In \cite{bib:constrained-knapsack}, the authors used a lagrangean relaxation on the precedence-constraint and the subgradient method to solve the problem faster then use a ``pegging'' test to guarantee optimality.


\subsection{0-1 Knapsack Problem}

\cite{bib:knapsack-problems} gives a definition for a \zoKPV (\zoKP):

\begin{eqnarray}
    \label{problem:knapsack-problem}
    \max & \displaystyle\sum\limits_{j=1}^{n} p_j x_j \\
    \subjectedTo
        & \displaystyle\sum\limits_{j=1}^{n} w_j x_j \leq c \nonumber\\
        & x_j \in \Set{0, 1} \quad \forall j \in \Set{1, \dots, n} \nonumber
\end{eqnarray}

in which $p_j$ and $w_j$ are know as the profit and the weight of the item $j$, respectively.

The problem proposed here is a knapsack problem adapted to satisfy two extra constraints: precedence-constrained and multi-dimensional weights. Besides, its profits are all one.
