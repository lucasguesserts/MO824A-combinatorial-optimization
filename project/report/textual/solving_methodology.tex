\subsection{GRASP + Tabu Search}

In order to find a time optimal algorithm to solve the multidimensional precedence-constrained knapsack problem (MPKP), we propose a comparative analysis of an implementation of a Integer Linear Program to solve MPKP and a near optimal algorithm utilizing GRASP and Tabu Search as GRASP's local search.

\begin{algorithm}[ht!]
    \caption{GRASP Pseudocode}
    \begin{algorithmic}[1]
        \Require{$MaxIterations, Seed$}
        \For{$k=1,\ldots,MaxIterations$}
            \State{$Solution \leftarrow Greedy Randomized Construction(Seed);$}
            \If{Solution is not feasible}
                \State{$Solution \leftarrow Repair(Solution)$;}
            \EndIf
            \State{$Solution \leftarrow LocalSearch(Solution);$}
            \State{$UpdateSolution(Solution, BestSolution);$}
        \EndFor
        \\\Return{$BestSolution$}
    \end{algorithmic}
\end{algorithm}
where the candidate list is build greedily utilizing the current maximal elements on our input graph(as long as they fit).
There's no need to use any repair method as the solution built from the candidate list will always fit in the knapsack.
That way, we do a local search and update our solution with the best one seen to far.

The trick here is to use a Tabu Search, instead of a naive local search method, to both avoid falling in local optima and utilizing the tabu list in order to avoid repeating moves and other. We use GRASP as a diversification strategy.

We now define exactly how this search is proposed.

Let $n \in \mathbb N$ be the number of binary variables of an input instance. The tabu list, let's call it $Tabu$ is a mapping $Tabu : StrN \rightarrow \mathbb N$, with pre-defined $Tabu.size$ capacity, where $StrN$ is an arrangement of $n$ bits let's call it $Arr$, where $n$ is the number of binary variables of our instance, and the $i$-th bit represents the state of $x_i$, where $Arr[i] = x_i$.

We also utilize a heap structure to get the earliest added tabu in $O(1)$ time.

During the local search, at each iteration we add the current state of our binary variables an element of $Tabu$. If the total number of tabus is equal to $Tabu.size$ we pop the earliest added tabu from our heap and remove that tabu from $Tabu$ while adding the new move to $Tabu$.

Each time that the Tabu Search tries to do a tabu move the integer in $Tabu$ associated with this move is looked up by the search, if the tabu exists, and if it equals $1$, that tabu is removed from the mapping.

Each move in the Tabu Search is a bit flip operation of a variable in a randomly selected, at every search iteration, subset $BinVars$ of the set of all binary variables of the input instance, such that $|BinVars| = k <= n$, where $k \in \mathbb{N}$ is a pre-defined parameter, as a way to limit the search space of the Tabu Search.

The best values for $k$ and $Tabu.size$ will be determined by tests.
