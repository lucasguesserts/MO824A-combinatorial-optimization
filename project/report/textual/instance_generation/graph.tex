\subsection{Graph Generation}

As stated earlier, the precedence contraint is defined by a Directed Acyclic Graph (DAG). We are actually going to prove a very interesting results:

\begin{theorem}
    Given an problem instance $I$ defined by a Directed Graph, it can be reduced to an instance $I'$ defined by a Directed Acyclic Graph Transitively Reduced \cite{bib:transitive-reduction}.
\end{theorem}

\begin{proof}
    It follows directly from \lemmaref{lemma:reduction-1} and \lemmaref{lemma:reduction-2}.
\end{proof}

\subsubsection{Removing cycles: Directed Graph to Directed Acyclic Graph}

\begin{lemma}
    \label{lemma:reduction-1}
    Given an problem instance $I$ defined by a Directed Graph, it can be reduced to an instance $I'$ defined by a Directed Acyclic Graph.
\end{lemma}

\begin{proof}
    Suppose that $I$ contains at most one cycle. Let:
    \begin{enumerate}
        \item a cycle $C = \Set{u_1, \dots, u_m} \subseteq \vertices$ defined by its vertices;
        \item $E_{in} = \SetOf{\tuple{u, v}}{v \in C}$ the edges that point to the vertices of the cycle $C$;
        \item $E_{out} = \SetOf{\tuple{u, v}}{u \in C}$ the edges that point from the vertices of the cycle $C$;
    \end{enumerate}
    Create a new graph replacing:
    \begin{enumerate}
        \item the cycle $C$ by a vertex $U$ with weight $\Sum{u \in C}{}{\weight_u}$;
        \item the edges $E_{in}$ by edges that point from the same vertices as originally to the vertex $U$;
        \item the edges $E_{out}$ by edges that point from $U$ to the same vertices as originally;
    \end{enumerate}
    Notice that if both \precedenceConstraint and \capacityConstraint are satisfied in $I$, then they are also satisfied in $I'$. Therefore, both instances are equivalent.
    For the general case in which $I$ has more than one cycle, one has to simply run the procedure described above for each cycle.
\end{proof}

\subsubsection{Transitive Reduction: Directed Acyclic Graph to Directed Acyclic Transitively Reduced Graph}

\begin{defn}
    The transitive reduction of a graph $G$ is the graph $G'$ which has as few edges as possible but the same transitive closure (reachability) of $G$ \cite{bib:transitive-reduction}.
\end{defn}

Obs: given a graph $G$ and its transitive reduction $G'$, I am refering to $G'$ as $G$ ``Transitively Reduced''.

\begin{lemma}
    \label{lemma:reduction-2}
    Given an problem instance $I$ defined by a Directed Acyclic Graph, it can be reduced to an instance $I'$ defined by a Directed Acyclic Graph Transitively Reduced.
\end{lemma}

\begin{proof}
    It follows directly from the fact that the transitive reduction preserves thetransitive closure (reachability).
\end{proof}

\subsubsection{How to generate a Directed Acyclic Transitively Reduced Graph}

Put simply:

\begin{enumerate}
    \item Generate a Directed Acyclic Graph Transitively Reduced by generating a Directed Acyclic Graph and computing one of its transitive reduction;
    \item Generate a Directed Acyclic Graph by generating a random upper triangular connectivity matrix (with only ones and zeros) with the main diagonal null (zero);
\end{enumerate}

There are several computational tools that implement the functionalities above. For this project, we used \cite{bib:numpy} for generating the random matrix and \cite{bib:networkx} for the graph manipulation.

\subsubsection{Graph Generation Parameters}

The following parameters are used to control the creation of the graph:

\begin{enumerate}
    \item number of nodes: integer positive number
    \item edge probability: a number between 0 and 1 (inclusive). It is the probability of each edge to exist. In some way, it controls the number of edges of the instance;
\end{enumerate}
