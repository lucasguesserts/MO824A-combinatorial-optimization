\section{Conclusion}

There are some hypothesis to explain the results observed:

\begin{enumerate}
    \item the problem is too easy;
    \item the problem is hard but the instances selected/generated are not;
    \item the strategies adopted are in fact a good fit for the problem;
\end{enumerate}

\subsection{Case 1}

Usually, in the literature, one considers the priced problem, i.e. the price of the items is not unitary. That seems to be a more difficult problem, and by not considering it in this project, it may have got too easy.

\subsection{Case II}

That doesn't seem to be the case. I have created and experimented with some values for the parameters of the instance generation and all of them seem to lead to the same results we got in this project. The instances generation method proposed also seems to be quite general and able to generate several different scenarios, it doesn't seem to be a problem

\subsection{Case III}

It is weird to see a simple algorithm such as Greedy to performa that well, consideting that more sophisticated approaches are available. But that demonstrates a good lesson: one has to attempt to solve the problem in order to better understand it and to learn how to solve it efficiently. The fancy Tabu algorithm could have been the best, or could not. The point is, it is a matter of experience and studying.

\subsection{Conclusion}

The algorithms proposed in this project demonstrated to be fit for the problem at hands. It is easy to solve in practice and a Greedy algorithm is the to-go algorithm in the majority of the cases. If an optimal solution is required, you can get one without the penalty of taking too long to run.
