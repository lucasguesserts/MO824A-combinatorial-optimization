\section{Definitions}

\subsection{Item}

An \textbf{Item} \itemO is a \tuple{6}:

\begin{equation}
	\label{definition:item}
	\itemO = \OrderedSet{
		\lx,
		\ly,
		\lz,
		\px,
		\py,
		\pz
	}
\end{equation}

in which the components represent the item's\footnote{See a definition for ``dimension'' in \cite{bib:dict-dimension}}:

\begin{enumerate}
	\item $\lx \in \realpositive$: dimension in the \px direction;
	\item $\ly \in \realpositive$: dimension in the \py direction;
	\item $\lz\in \realpositive$: dimension in the \pz direction;
	\item $\px \in \realnonnegative$: x position;
	\item $\py \in \realnonnegative$: y position;
	\item $\pz \in \realnonnegative$: z position;
\end{enumerate}

We represent by $\itemSet = \itemDomain$ the set of all items.

The reason an items is seen in that way is because of the stacking

%\subsection{Container}
%
%A \textbf{Container} \containero is a \tuple{3}:
%
%\begin{equation}
%	\label{definition:container}
%	\containero = \OrderedSet{
%		\lx,
%		\ly,
%		\lz
%	}
%\end{equation}
%
%in which the components represent the container's:
%
%\begin{enumerate}
%	\item $\lx \in \realpositive$: dimension in the \px direction (the length);
%	\item $\ly \in \realpositive$: dimension in the \py direction (the width);
%	\item $\lz \in \realpositive$: dimension in the \pz direction (the height);
%\end{enumerate}

\subsection{Vehicle}

A Vehicle \vehicleO is a \tuple{3}:

\begin{equation}
	\label{definition:vehicle}
	\vehicleO = \OrderedSet{
		\nAxles,
		\loadingFunction,
		\loadingLimit
	}
\end{equation}

in which:

\begin{enumerate}
	\item $\nAxles$ is the number of components of the vehicle's loadings;
	\item $\function{\loadingFunction}{\itemSet}{\loadingCodomain}$: a function that associates every item to a vehicle's loading;
	\item $\loadingLimit \in \loadingCodomain$: represents the vehicle's loading limit;
\end{enumerate}

We represent by $\vehicleSet$ the set of all vehicles.
