\documentclass{article}
\usepackage[a4paper]{geometry}
\usepackage[brazil]{babel}
\usepackage[utf8]{inputenc}
\usepackage{url}
\usepackage{hyperref}
\usepackage{graphicx}
\usepackage{amssymb}
\usepackage{amsmath}
\usepackage{amsfonts}
\usepackage{xspace}
\usepackage{yfonts}
\usepackage{mathrsfs}

\title{Computational Project - Theme Proposal}
\author{
	Luiz Fernando Bueno Rosa - RA: 221197 \\
	Lucas Guesser Targino da Silva - RA: 203534
}

\newcommand{\real}{\ensuremath{\mathbb{R}}\xspace}
\newcommand{\realnonnegative}{\ensuremath{\mathbb{R}_{+}}\xspace}
\newcommand{\realpositive}{\ensuremath{\mathbb{R}_{+}^{*}}\xspace}
\newcommand{\nrealnonnegative}[1]{\ensuremath{\mathbb{R}_{+}^{#1}}\xspace}
\newcommand{\nrealpositive}[1]{\ensuremath{\mathbb{R}_{+}^{*#1}}\xspace}

\newcommand{\true}{\ensuremath{true}\xspace}
\newcommand{\false}{\ensuremath{false}\xspace}

\newcommand{\abs}[1]{\ensuremath{\left| #1 \right|}\xspace}
\newcommand{\tuple}[1]{#1-tuple\xspace}
\newcommand{\Set}[1]{\ensuremath{\left\{#1\right\}}}
\newcommand{\SetOf}[2]{\ensuremath{\left\{ #1 : #2 \right\}}\xspace}
\newcommand{\OrderedSet}[1]{\ensuremath{\langle#1\rangle}\xspace}
\newcommand{\function}[3]{\ensuremath{#1: #2 \rightarrow #3}\xspace}

\newcommand{\bigo}[1]{\ensuremath{\mathcal{O}\left( #1 \right)}}

\newcommand{\vehicleO}{\ensuremath{\upsilon}\xspace}
\newcommand{\vehicleSet}{\mathcal{V}\xspace}
\newcommand{\loadingFunction}{\ensuremath{\eta}\xspace}
\newcommand{\loadingFunctionApply}[1]{\loadingFunction \left( #1 \right)\xspace}
\newcommand{\loadingLimit}{\ensuremath{L}\xspace}
\newcommand{\nAxles}{\ensuremath{\alpha}\xspace}
\newcommand{\loadingCodomain}{\nrealnonnegative{\nAxles}}

\newcommand{\itemO}{\ensuremath{\iota}\xspace}
\newcommand{\itemSet}{\ensuremath{\mathcal{I}}\xspace}
\newcommand{\itemDomain}{\ensuremath{\nrealpositive{3} \times \nrealnonnegative{3}}\xspace}

\newcommand{\containero}{\ensuremath{c}\xspace}
\newcommand{\containerSet}{\ensuremath{\mathcal{C}}\xspace}

\newcommand{\lx}{\ensuremath{\chi}\xspace}
\newcommand{\ly}{\ensuremath{\psi}\xspace}
\newcommand{\lz}{\ensuremath{\omega}\xspace}
\newcommand{\px}{\ensuremath{x}\xspace}
\newcommand{\py}{\ensuremath{y}\xspace}
\newcommand{\pz}{\ensuremath{z}\xspace}

\newcommand{\itemInput}{\ensuremath{\mathit{I}_{o}}\xspace}
\newcommand{\itemOutput}{\ensuremath{\mathit{I}_{f}}\xspace}
\newcommand{\vehicleInput}{\vehicleO}
\newcommand{\constraintsPredicateSymbol}{\ensuremath{\mathscr{C}}\xspace}
\newcommand{\constraintsPredicate}[1]{\ensuremath{\constraintsPredicateSymbol \left( #1 \right)}\xspace}

\begin{document}

\maketitle

\section{Definitions}

\subsection{Item}

An \textbf{Item} \itemO is a \tuple{6}:

\begin{equation}
	\label{def:item}
	\itemO = \OrderedSet{
		\lx,
		\ly,
		\lz,
		\px,
		\py,
		\pz
	}
\end{equation}

in which the components represent the item's\footnote{See a definition for ``dimension'' in \cite{bib:dict-dimension}}:

\begin{enumerate}
	\item $\lx \in \realpositive$: dimension in the \px direction;
	\item $\ly \in \realpositive$: dimension in the \py direction;
	\item $\lz\in \realpositive$: dimension in the \pz direction;
	\item $\px \in \realnonnegative$: x position;
	\item $\py \in \realnonnegative$: y position;
	\item $\pz \in \realnonnegative$: z position;
\end{enumerate}

We represent by $\itemSet = \itemDomain$ the set of all items.

%\subsection{Container}
%
%A \textbf{Container} \containero is a \tuple{3}:
%
%\begin{equation}
%	\label{def:container}
%	\containero = \OrderedSet{
%		\lx,
%		\ly,
%		\lz
%	}
%\end{equation}
%
%in which the components represent the container's:
%
%\begin{enumerate}
%	\item $\lx \in \realpositive$: dimension in the \px direction (the length);
%	\item $\ly \in \realpositive$: dimension in the \py direction (the width);
%	\item $\lz \in \realpositive$: dimension in the \pz direction (the height);
%\end{enumerate}

\subsection{Vehicle}

A Vehicle \vehicleO is a \tuple{3}:

\begin{equation}
	\label{def:vehicle}
	\vehicleO = \OrderedSet{
		\nAxles,
		\loadingFunction,
		\loadingLimit
	}
\end{equation}

in which:

\begin{enumerate}
	\item $\nAxles$ is the number of components of the vehicle's loadings;
	\item $\function{\loadingFunction}{\itemSet}{\loadingCodomain}$: a function that associates every item to a vehicle's loading;
	\item $\loadingLimit \in \loadingCodomain$: represents the vehicle's loading limit;
\end{enumerate}

We represent by $\vehicleSet$ the set of all vehicles.

\section{Problem Statement}

\subsection{Input}

\begin{enumerate}
	\item $\itemInput \subseteq \itemSet$: the set of items
	\item $\vehicleInput \in \vehicleSet$: the vehicle
\end{enumerate}

\subsection{Constraints}

\subsubsection*{Loading Limit Constraint}
	\label{constraint:loading-limit}
\begin{equation}
	\displaystyle\sum\limits_{\itemO \in \itemInput}
		\loadingFunctionApply{\itemO}
		\leq \loadingLimit
\end{equation}

\subsubsection*{Stacking Constraint}
	\label{constraint:stacking}
\begin{equation}
	\mbox{An item can only be removed if all items above it have already been removed}
\end{equation}

\subsubsection*{Constraints Predicate}

Given an $\itemOutput \subseteq \itemInput$, we represent by $\constraintsPredicate{\itemOutput} \in \Set{\true, \false}$ whether \itemOutput satisfy all the above constraints or not.

\subsection{Output}

A subset $\itemOutput \subseteq \itemInput$ of the input items.

\subsection{Objective}

\begin{equation}
	min\SetOf{\abs{\itemInput} - \abs{\itemOutput}}{\itemOutput \subseteq \itemInput \ \land \ \constraintsPredicate{\itemOutput}}
\end{equation}

Minimize the number of items removed so that all constraints are satisfied.

% ----------- REFERENCES -----------
\bibliographystyle{unsrt}
\bibliography{bibliography}

\end{document}
