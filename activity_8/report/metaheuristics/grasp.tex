\subsection{\grasp}
\label{subsection:grasp}

A metaheurística \grasp é descrita no \aref{algorithm:grasp}, sua estratégia construtiva em \aref{algorithm:grasp-construction}, e sua estratégia de busca local em \aref{algorithm:grasp-local-search}, todos descritos no \appendixref{appendix:grasp}.

Ambas usam a estratégia construtiva padrão (\aref{algorithm:grasp-construction}) e o parâmetro $\alpha$ com valor $0.2$.

Elas diferem na estratégia de busca local, entretanto. A primeira variação, chamada \graspBest, utiliza \bestImproving, em que toda a vizinhança é percorrida e a melhor opção selecionada. A segunda variação, chamada \graspFirst, utiliza \firstImproving, em que a busca na vizinhança retorna a primeira solução encontrada que seja melhor do que a atual.

Em ambas, \bestImproving e \firstImproving, a vizinhança é definida como o conjunto de soluções obtidas a partir da solução atual que tenham:

\begin{enumerate}
    \item 1 elemento a mais - adição;
    \item 1 elemento a menos - remoção;
    \item 1 elemento trocado (equivalente a uma adição e uma remoção) - troca;
\end{enumerate}

\subsubsection{\graspBest}
\label{subsubsection:grasp-best}

\begin{enumerate}
    \item Estratégia construtiva: \textbf{padrão} com $\alpha$ igual a \textbf{0.2}
    \item Estratégia de busca local: \textbf{best improving}
\end{enumerate}

\subsubsection{\graspFirst}
\label{subsubsection:grasp-first}

\begin{enumerate}
    \item Estratégia construtiva: \textbf{padrão} com $\alpha$ igual a \textbf{0.2}
    \item Estratégia de busca local: \textbf{first improving}
\end{enumerate}
