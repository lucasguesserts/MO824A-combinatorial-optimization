\subsection{\genetic}
\label{subsection:genetic}

A metaheurística \tabu é descrita no \aref{algorithm:genetic}.

Ambas as variações descritas nessa subseção usam:

\begin{enumerate}
    \item população inicial aleatória;
    \item tamanho da população igual a 100;
    \item seleção para reprodução em torneio: dois cromossomos são escolhidos aleatoriamente e o
    melhor dos dois (em relação à função de aptidão) é escolhido para continuar enquanto o outro é
    descartado.
    \item reprodução com \textit{two point crossover (2X)};
    \item Critério de parada: 1000 gerações;
\end{enumerate}

\subsubsection{\geneticVanilla}

A primeira variação, chamada \geneticVanilla, implementa \genetic conforme descrito na \ssref{subsection:genetic} com as seguintes adições/modificações:

\begin{enumerate}
    \item taxa de mutação igual a $0.5\%$;
    \item seleção da nova população: descendentes, substituindo-se o pior deles pelo melhor gene conhecido
\end{enumerate}

\subsubsection{\geneticSteady}

A segunda variação, chamada \geneticSteady, implementa \genetic conforme descrito na \ssref{subsection:genetic} com as seguintes adições/modificações:

\begin{enumerate}
    \item taxa de mutação igual a $1\%$;
    \item seleção da nova população: os 100 melhores genes entre mães e filhos (conhecida como \textit{Steady-State $\lambda + \mu$});
\end{enumerate}
