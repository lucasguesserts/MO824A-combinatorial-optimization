\subsection{\tabu}
\label{subsection:tabu}

A metaheurística \tabu é descrita no \aref{algorithm:tabu}.

Ambas as variações descritas nessa subseção usam:

\begin{enumerate}
    \item solução inicial a solução da estratégia construtiva do \grasp, \aref{algorithm:grasp-construction};
    \item busca local \bestImproving. Ela é similar a \bestImproving descrita na \ssref{subsection:grasp}, entretanto ela não considera a adição ou remoção de elementos que estão na lista tabu $T$ (a menos que eles levem a uma solução melhor do que $S^*$);
    \item \tenureRatio, parâmetro que controla o tamanho da lista tabu $T$, igual a $0.4$, o que significa que $T$ pode ter tamanho até 40\% do tamanho da entrada.
\end{enumerate}

A primeira variação, chamada \tabuVanilla, implementa \tabu com as características acima. A segunda variação, chamada \tabuMod, inclui estratégias de diversificação e intensificação, descritas nas \sssref{subsubsection:tabu-intensification} e \sssref{subsubsection:tabu-diversification}.

\subsubsection{Estratégia de Intensificação}
\label{subsubsection:tabu-intensification}

A estratégia de intensificação aumenta o tamanho da vizinhaça, ao invés de considerar 1 adição, 1 remoção e 1 troca, ela considera:

\begin{enumerate}
    \item 2 adições e 1 remoção;
    \item 2 remoções e 1 adição;
    \item 2 adições e 2 remoções;
\end{enumerate}

A intensificação é feita em torno da melhor solução conhecida. Ela é ativada quando passam-se muitas iterações (o critério de parada é definido como um número máximo de iterações, e ``muitas iterações'' significa 20\% número máximo de iterações) sem melhora na solução ótima e sem sua ativação.

\subsubsection{Estratégia de Diversificação}
\label{subsubsection:tabu-diversification}

A estratégia de diversificação constroi uma nova solução, utilizando a estratégia construtiva do GRASP, e recomeça a busca nesse outro local do espaço de solução. Além disso, antes de começar a busca, faz-se uma busca intensiva (usando a estratégia de intensificação) em torno dessa nova solução.

Seu critério de ativação é o mesmo da Estratégia de Intensificação, excetuando-se que Diversificação é ativada apenas quando Intensificação não é (no programa, há um registro separado para quando cada uma delas foi ativada pela última vez).

Assim, quando se passaram muitas iterações sem melhora na solução ótima, primeiro tenta-se intensificação. Caso essa falhe, executa-se a diversificação.

\subsubsection{\tabuVanilla}
\label{subsubsection:tabu-vanilla}

\begin{enumerate}
    \item Solução inicial: saída da estratégia construtiva do \grasp
    \item \tenureRatio: \textbf{0.4}
    \item Estratégia de busca local: \textbf{\bestImproving}
\end{enumerate}

\subsubsection{\tabuMod}
\label{subsubsection:tabu-mod}

\begin{enumerate}
    \item Solução inicial: saída da estratégia construtiva do \grasp
    \item \tenureRatio: \textbf{0.4}
    \item Estratégia de busca local: \textbf{\bestImproving}
    \item Adição das estratégias: \textbf{Intensificação e Diversificação}
\end{enumerate}
