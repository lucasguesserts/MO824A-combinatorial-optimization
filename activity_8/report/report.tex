\documentclass[11pt]{article}
\usepackage[utf8]{inputenc}
\usepackage[T1]{fontenc}
\usepackage[brazilian]{babel}
\usepackage{graphicx}
\usepackage{longtable}
\usepackage{wrapfig}
\usepackage{rotating}
\usepackage[normalem]{ulem}
\usepackage{amsmath}
\usepackage{amsfonts}
\usepackage{amssymb}
\usepackage{amsthm}
\usepackage{capt-of}
\usepackage{hyperref}
\usepackage{geometry}
\usepackage{booktabs}
\usepackage{url}
\usepackage{hyperref}
\usepackage{multicol}
\usepackage{algorithm}
\usepackage{algorithmicx}
\usepackage[noend]{algpseudocode}
\usepackage[title]{appendix}
\usepackage{float}
\usepackage{xspace}

\theoremstyle{definition}
\newtheorem{defn}{Definição}
\newtheorem{fact}{Fato}

\newcommand{\Set}[1]{\left\{#1\right\}}
\newcommand{\Sum}[4]{\displaystyle\sum\limits_{#1 = #2}^{#3} #4}
\newcommand{\function}[3]{#1: #2 \rightarrow #3}
\newcommand{\transpose}[1]{#1^{T}}

\newcommand{\B}{\mathbb{B}}
\newcommand{\Bn}{\mathbb{B}^n}
\newcommand{\R}{\mathbb{R}}
\newcommand{\Rn}{\mathbb{R}^n}
\newcommand{\Z}{\mathbb{Z}}
\newcommand{\Zn}{\mathbb{Z}^n}

\newcommand{\qbf}{QBF\xspace}
\newcommand{\maxqbf}{MAX-QBF\xspace}
\newcommand{\maxkqbf}{MAX-KQBF\xspace}
\newcommand{\maxkqbffull}{\textit{MAX-QBF com mochila}\xspace}

\newcommand{\w}{w}
\newcommand{\W}{W}

\newcommand{\aij}{a_{i,j}}
\newcommand{\A}{A}

\newcommand{\grasp}{\textit{GRASP}\xspace}
\newcommand{\graspfull}{\textit{Greedy Randomized Adaptive Search Procedure}\xspace}

\newcommand{\tabu}{\textit{Tabu}\xspace}
\newcommand{\tabufull}{\textit{Tabu Search}\xspace}

\newcommand{\genetic}{\textit{GA}\xspace}
\newcommand{\geneticfull}{\textit{Genetic Algorithm}\xspace}

\newcommand{\aref}[1]{Apêndice \ref{#1}}

\geometry{a4paper, left=20mm, top=35mm, bottom=35mm, right=20mm}

\author{Lucas Guesser Targino da Silva (203534)}
\date{\today}
\title{MO824 - Análise Comparativa entre as metaheurísticas \grasp, \tabu, e \genetic para a resolução do problema MAX-QBF com mochila}

\begin{document}

\maketitle

Esse trabalho tem como objetivo comparar os resultados obtidos pelas implementações de três metaheurísticas:

\begin{enumerate}
    \item \graspfull (\grasp) \cite{bib:grasp}
    \item \tabufull (\tabu) \cite{bib:tabu}
    \item \geneticfull (genetic) \cite{bib:genetic}
\end{enumerate}

O problema resolvido foi o \maxkqbffull, descrito no \aref{appendix:max-kqbf}.


\bibliographystyle{ieeetr}
\bibliography{bibliography}

\begin{appendices}

\section{\maxkqbffull (\maxkqbf)}
\label{appendix:max-kqbf}

\begin{defn}[Conjunto Binário]
    $\B = \Set{0, 1}$
\end{defn}

\begin{defn}[Função Binária Quadrática (\qbf)]
    É uma função $\function{f}{\Bn}{\Z}$ da forma:
    $$
        f(x)
        = \Sum{j}{1}{n}{x_i \cdot \aij \cdot x_j}
        = \transpose{x} \cdot \A \cdot x
    $$
    em que $\aij \in \Z, \ \forall i,j \in \Set{1, \cdots, n}$ e $\A$ é a matriz $n$ por $n$ induzida pelos $\aij$.
\end{defn}

\begin{defn}[Problema de Maximização de uma Função Binária Quadrática (\maxqbf)]
Dada uma \qbf $f$, um \maxqbf é um problema da forma:
$$
    \max\limits_{x} f(x)
$$
\end{defn}

\begin{fact}
\maxqbf é NP-difícil \cite{bib:qbf}
\end{fact}

\begin{defn}[Maximum knapsack quadractic binary function (\maxkqbf)]
Dada uma \qbf $f$, um vetor $\w \in \Zn$\footnote{O problema original foi definido com números reais. Decidimos aqui utilizar inteiros por dois motivos. Primeiro, todas as instâncias fornecidas possuem apenas valores inteiros para $\aij, \w, \W$. Garante-se que os valores são sempre inteiros pois $\Z$ é fechado nas operações envolvidas: adição e multiplicação. Segundo, simplifica a implementação e comparações (não é necessário fazer comparação de números em ponto flutuante).}, e um valor $\W \in \Z$, um \maxkqbf é um problema da forma:
\begin{eqnarray*}
    \max & f(x) \\
    \mbox{subjected to} & \transpose{\w} x \leq \W \\
    & x \in \Bn
\end{eqnarray*}
\end{defn}

\section{Implementação e execução dos experimentos}

O programs foram executados num ideapad S145 81S90005BR: Lenovo IdeaPad S145 Notebook Intel Core i5-8265U (6MB Cache, 1.6GHz, 8 cores), 8GB DDR4-SDRAM, 460 GB SSD, Intel UHD Graphics 620 no ambiente:

\begin{enumerate}
    \item sistema operacional: Fedora 35
    \item Java versão 17
    \item Gradle versão 7.4
\end{enumerate}

O desenvolvimento da solução do problema foi feito em Java, baseado nos frameworks disponibilizados pelos professores. O código pode ser encontrado em \cite{bib:github}.


\end{appendices}

\end{document}
