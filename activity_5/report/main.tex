\documentclass[brazilian,12pt]{article}
\usepackage[portuguese]{babel}
\usepackage[utf8]{inputenc}
\usepackage{graphicx} % Allows you to insert figures
\usepackage{float} % posição de figura ([H])
\usepackage{amsmath} % Allows you to do equations
\usepackage{fancyhdr} % Formats the header
\usepackage{geometry} % Formats the paper size, orientation, and margins
\usepackage[bitstream-charter]{mathdesign} % Mudamos a fonte para Charter BT
\usepackage[T1]{fontenc} % Mudamos a fonte para Charter BT
\usepackage{tablefootnote}
\usepackage{lscape}
\usepackage{booktabs}
\usepackage{paralist} % compactitem
\usepackage[pdftex]{hyperref}
\usepackage{mathtools}
\newcommand{\ceil}[1]{\left\lceil #1 \right\rceil}
\usepackage{scalefnt}
\usepackage{hyperref}
\hypersetup{
    colorlinks=true,
    linkcolor=blue,
    filecolor=magenta,      
    urlcolor=blue,
    pdftitle={Overleaf Example},
    pdfpagemode=FullScreen,
}
\usepackage{cleveref} % referência com nomes

% \usepackage{import}

\urlstyle{same}
\linespread{1.25} % about 1.5 spacing in Word
\setlength{\parindent}{0pt} % no paragraph indents
\setlength{\parskip}{1em} % paragraphs separated by one line

\renewcommand{\headrulewidth}{0pt}
\geometry{letterpaper, portrait, margin=1in}
\setlength{\headheight}{14.49998pt}

\newcommand\titleofdoc{Aplicação da \textit{Tabu Search} no Problema MAX-KBQF} %%%%% Put your document title in this argument
\newcommand\GroupName{Grupo 7} %%%%% Put your group name here. If you are the only member of the group, just put your name

\begin{document}
\begin{titlepage}
   \begin{center}
        \vspace*{3cm} % Adjust spacings to ensure the title page is generally filled with text

        % \Huge{\titleofdoc} 

        % \vspace{0.5cm}
        \LARGE{Aplicação da \textit{Tabu Search} no Problema MAX-KBQF}
        \Large{Atividade 5}
        
            
        \vspace{2 cm}
        \Large{\GroupName}
       
        \vspace{0.25cm}
        %% https://www.overleaf.com/project/6250b01f16f6860fd0884dcd
        \large{Ítalo Fernandes Gonçalves - RA: 234990}\\
        \large{Lucas Guesser Targino da Silva - RA: 203534}\\
        \large{Luiz Gustavo Silva Aguiar - RA: 240499}\\
       
        \vspace{8 cm}
        \Large{Maio de 2022}
        
        \vspace{0.25 cm}
        \Large{MC859/MO824}
       \vfill
    \end{center}
\end{titlepage}

\setcounter{page}{1}
\pagestyle{fancy}
\fancyhf{}
\rhead{\thepage}
\lhead{\GroupName - \titleofdoc}


\section{Definição do Problema}

Neste atividade, vamos abordar o problema denominado por \textit{Maximum Knapsack quadratic binary function} (MAX-KQBF). No MAX-KQBF temos uma mochila de capacidade W e um conjunto de itens I, onde cada item $i \in I$ tem um peso $w_i \in R$. Desejamos empacotar os itens $I$ na mochila, de forma a maximizar uma função quadrática e respeitando a capacidade da mochila, isto é, a soma dos pesos dos itens colocados na mochila precisa ser menor ou igual a $W$. Este problema pode ser formulado da seguinte forma:

\begin{align}
     &Max \quad & \displaystyle{\sum_{i=1}^{n} \sum_{j=1}^{n} a_{ij} \cdot x_i \cdot x_j} \label{obj}\\
     &Sujeito \mbox{ a}       & \displaystyle{\sum_{i=1}^{n} w_i \cdot x_i\leq \textit{W}}&\quad&\\
     &&x_i \in \{0,1\} \quad && \forall i = \{1,\dots,n\}&
\end{align}

onde $a_{ij}$, $w_i$ e $W \in \mathbb{R}$ são parâmetros do problema, e $n = |I|$. O MAX-KQBF é uma generalização do problema \textit{Maximum Quadratic Binary Function} (MAXBQF), no qual é um problema NP-difícil \cite{NPHARD}

\section{\textit{Tabu Search}}

A metaheurística \textit{Tabu Search} (TS) é um procedimento adaptativo auxiliar, que direciona um algoritmo de busca local na exploração contínua dentro de um espaço de busca. Diferente de outras abordagens, a TS evita que retornar a um ótimo local visitado previamente de forma a superar a otimalidade local \cite{ts}.

Nessa abordagem, o principio básico parte da ideia de direcionar a busca local, permitindo movimentos sem melhoria. O retorno as soluções visitadas previamente é evitado com uma memorização através das chamadas listas tabu. Essas listas armazenam o histórico recente da busca. Sua implementação consiste basicamente na definição de uma busca local e na estrategia de memorização das listas tabu. Tais etapas serão descritas sobre a ótica do problema MAX-KQBF, adentrando na adequação da abordagem ao problema. 

\subsection{Solução inicial}
  Utilizamos a heurística construtiva do método GRASP para encontrarmos uma solução inicial para o problema. A solução foi construída a partir de uma iteração do GRASP e ela foi construída de forma puramente gulosa tomando assim $ \alpha = 0$.

\subsection{Lista Tabu}
    A lista tabu que é uma das principais características dessa abordagem, evitam que a busca seja cíclica, voltando em soluções previamente exploradas e ainda colaborando para a realização de uma busca mais densa dentro do espaço de busca. Cada passo da busca é registrado na lista segundo as regras de ativação. Depois de adicionados permanecem na lista por um determinado período de iterações controlado pelo parâmetro \textit{tabu tenure}. Nas implementações realizadas, esse valor foi definido como uma razão do tamanho $n$ da entrada, e foram utilizados os valores $0.2 \cdot n$ e $0.4 \cdot n$.
    
\subsection{Critério de aspiração} 

    Os tabus, como são chamadas as memorizações realizadas, são bem decisivas pois podem proibir movimentos atraentes mesmo quando não há possibilidade de ciclos ou podem levar a estagnação geral no processo de busca. O critério de aspiração permitem cancelar memorizações realizadas. Para a atual abordagem ao problema MAX-KQBF foi utilizado o critério de aspiração mais simples e utilizado, que consistem em permitir o movimento, mesmo que seja tabu, se resultar em uma solução com valor objetivo melhor que o da melhor solução conhecida.

\subsection{Operadores de busca}

A heurística de busca local com a estratégia \textit{Best Improvement} consiste em analisar todas as soluções vizinhas e se mover para aquela que tiver a melhor avaliação e que represente uma melhora em relação à solução corrente. Se não houver solução de melhora, então o método para e retorna a solução corrente como ótimo local em relação à estrutura de vizinhança utilizada. Diferente do método anterior, a \textit{Fisrt improvement}, a decisão para mover-se para uma nova solução vizinha não ocorre após a análise de toda a vizinhança mas sim na primeira solução vizinha encontrada melhor que a solução atual.

% \subsection{Critério de parada}

\subsection{Estrategias tabu alternativas}
    Foram utilizadas 2 estrategias tabu alternativas: \textit{diversification by restart} e \textit{intensification by restart}.
    A estrategia \textit{Diversification by restart} constroi uma nova solução (aleatorizada) e recomeça o processo de busca. A ideia dela é explorar regiões não atingidas pelo caminho até então percorrido. Já a estrategia \textit{intensification by restart} faz uma busca intensa (isto é, com uma vizinhança maior) na vizinhança de uma solução interessante. No caso presente, utilizou-se a a melhor solução conhecida como reinício e considerou-se a vizinhança para a busca intensiva:
    
\begin{itemize}
    \item inserção de um elemento e remoção de dois;
    \item remoção de um elemento e inserção de dois;
    \item inserção de dois elementos e remoção de dois\footnote{essa opção foi desabilitada nas instâncias de tamanho 400 pois consumiam muito tempo computacional.};
\end{itemize}

\section{Resultados Computacionais}

    A implementação foi feita com base na \textit{framework} disponibilizada com a implementação base da \textit{Tabu Search} na linguagem de programação java. O problema foi executado num ideapad S145 81S90005BR: Lenovo IdeaPad S145 Notebook Intel Core i5-8265U (6MB Cache, 1.6GHz), 8GB DDR4-SDRAM, 460 GB SSD, Intel UHD Graphics 620. O sistema operacional foi o Fedora 35 executando o Java 17 e Gradle 7. O código desenvolvido pode ser encontrado em \cite{Github}.
    
    As tabelas \ref{table:all-data-0}, \ref{table:all-data-1}, \ref{table:all-data-2}, \ref{table:all-data-3}, \ref{table:all-data-4}, \ref{table:all-data-5} e \ref{table:all-data-6} tem os resultados obtidos nos experimentos, variando as variações conforme solicitado nos requisitos desta atividade. A primeira coluna enumera cada instancia, a segunda coluna tem a identificação da instância atual, a terceira coluna a definição da busca local utilizada, a quarta coluna a variação do método e por fim na quinta e sexta coluna o tempo de execução e a melhor solução encontrada. Vale ressaltar que na variação do método, \textit{default} significa que não foi utilizada nenhuma estratégia de intensificação e diversificação, \textit{intensification} significa que foi utilizada apenas \textit{intensification by restart}, e \textit{diversification} significa que foram utilizadas \textit{intensification and diversification by restart}.

\begin{landscape}

\begin{table}
\centering
\begin{tabular}{lllllll}
\toprule
{} & instances & local search & tenure ratio & method variation & running time & best cost \\
\midrule
0  &       020 &         best &          0.2 &          default &     1.263000 &        93 \\
1  &       020 &         best &          0.2 &  intensification &     0.447000 &        93 \\
2  &       020 &         best &          0.2 &  diversification &     0.246000 &        93 \\
3  &       020 &         best &          0.4 &          default &     0.997000 &       120 \\
4  &       020 &         best &          0.4 &  intensification &     0.476000 &       120 \\
5  &       020 &         best &          0.4 &  diversification &     0.255000 &       120 \\
6  &       020 &        first &          0.2 &          default &     0.043000 &        93 \\
7  &       020 &        first &          0.2 &  intensification &     0.035000 &       104 \\
8  &       020 &        first &          0.2 &  diversification &     0.031000 &        93 \\
9  &       020 &        first &          0.4 &          default &     0.057000 &       102 \\
10 &       020 &        first &          0.4 &  intensification &     0.205000 &       110 \\
11 &       020 &        first &          0.4 &  diversification &     0.035000 &       104 \\
\bottomrule
\end{tabular}
\caption{Solução obtida para cada configuração e instância do problema - parte 0.}
\label{table:all-data-0}
\end{table}

\begin{table}
\centering
\begin{tabular}{lllllll}
\toprule
{} & instances & local search & tenure ratio & method variation & running time & best cost \\
\midrule
12 &       040 &         best &          0.2 &          default &     2.660000 &       239 \\
13 &       040 &         best &          0.2 &  intensification &     1.633000 &       290 \\
14 &       040 &         best &          0.2 &  diversification &     1.284000 &       260 \\
15 &       040 &         best &          0.4 &          default &     2.707000 &       308 \\
16 &       040 &         best &          0.4 &  intensification &     4.774000 &       316 \\
17 &       040 &         best &          0.4 &  diversification &     1.218000 &       303 \\
18 &       040 &        first &          0.2 &          default &     0.049000 &       201 \\
19 &       040 &        first &          0.2 &  intensification &     0.090000 &       239 \\
20 &       040 &        first &          0.2 &  diversification &     0.195000 &       243 \\
21 &       040 &        first &          0.4 &          default &     0.114000 &       239 \\
22 &       040 &        first &          0.4 &  intensification &     0.147000 &       239 \\
23 &       040 &        first &          0.4 &  diversification &     0.224000 &       239 \\
\bottomrule
\end{tabular}
\caption{Solução obtida para cada configuração e instância do problema - parte 1.}
\label{table:all-data-1}
\end{table}

\begin{table}
\centering
\begin{tabular}{lllllll}
\toprule
{} & instances & local search & tenure ratio & method variation & running time & best cost \\
\midrule
24 &       060 &         best &          0.2 &          default &     5.390000 &       368 \\
25 &       060 &         best &          0.2 &  intensification &     4.510000 &       480 \\
26 &       060 &         best &          0.2 &  diversification &     3.904000 &       483 \\
27 &       060 &         best &          0.4 &          default &     5.427000 &       491 \\
28 &       060 &         best &          0.4 &  intensification &     5.166000 &       446 \\
29 &       060 &         best &          0.4 &  diversification &     5.662000 &       491 \\
30 &       060 &        first &          0.2 &          default &     0.066000 &       413 \\
31 &       060 &        first &          0.2 &  intensification &     0.238000 &       452 \\
32 &       060 &        first &          0.2 &  diversification &     0.715000 &       406 \\
33 &       060 &        first &          0.4 &          default &     0.163000 &       408 \\
34 &       060 &        first &          0.4 &  intensification &     0.327000 &       397 \\
35 &       060 &        first &          0.4 &  diversification &     3.064000 &       455 \\
\bottomrule
\end{tabular}
\caption{Solução obtida para cada configuração e instância do problema - parte 2.}
\label{table:all-data-2}
\end{table}

\begin{table}
\centering
\begin{tabular}{lllllll}
\toprule
{} & instances & local search & tenure ratio & method variation & running time & best cost \\
\midrule
36 &       080 &         best &          0.2 &          default &    10.182000 &       683 \\
37 &       080 &         best &          0.2 &  intensification &     7.984000 &       662 \\
38 &       080 &         best &          0.2 &  diversification &     7.651000 &       780 \\
39 &       080 &         best &          0.4 &          default &    10.318000 &       783 \\
40 &       080 &         best &          0.4 &  intensification &     8.769000 &       732 \\
41 &       080 &         best &          0.4 &  diversification &     6.590000 &       702 \\
42 &       080 &        first &          0.2 &          default &     0.089000 &       667 \\
43 &       080 &        first &          0.2 &  intensification &     0.513000 &       674 \\
44 &       080 &        first &          0.2 &  diversification &     2.313000 &       680 \\
45 &       080 &        first &          0.4 &          default &     0.273000 &       692 \\
46 &       080 &        first &          0.4 &  intensification &     0.649000 &       692 \\
47 &       080 &        first &          0.4 &  diversification &     2.205000 &       692 \\
\bottomrule
\end{tabular}
\caption{Solução obtida para cada configuração e instância do problema - parte 3.}
\label{table:all-data-3}
\end{table}

\begin{table}
\centering
\begin{tabular}{lllllll}
\toprule
{} & instances & local search & tenure ratio & method variation & running time & best cost \\
\midrule
48 &       100 &         best &          0.2 &          default &    17.558000 &      1220 \\
49 &       100 &         best &          0.2 &  intensification &    15.455000 &      1116 \\
50 &       100 &         best &          0.2 &  diversification &   130.585000 &      1193 \\
51 &       100 &         best &          0.4 &          default &    17.549000 &      1227 \\
52 &       100 &         best &          0.4 &  intensification &    28.078000 &      1249 \\
53 &       100 &         best &          0.4 &  diversification &    17.294000 &      1225 \\
54 &       100 &        first &          0.2 &          default &     0.118000 &       979 \\
55 &       100 &        first &          0.2 &  intensification &     1.041000 &       966 \\
56 &       100 &        first &          0.2 &  diversification &     5.663000 &       850 \\
57 &       100 &        first &          0.4 &          default &     0.350000 &      1121 \\
58 &       100 &        first &          0.4 &  intensification &     1.519000 &      1128 \\
59 &       100 &        first &          0.4 &  diversification &     6.309000 &       878 \\
\bottomrule
\end{tabular}
\caption{Solução obtida para cada configuração e instância do problema - parte 4.}
\label{table:all-data-4}
\end{table}

\begin{table}
\centering
\begin{tabular}{lllllll}
\toprule
{} & instances & local search & tenure ratio & method variation & running time & best cost \\
\midrule
60 &       200 &         best &          0.2 &          default &   123.439000 &      3947 \\
61 &       200 &         best &          0.2 &  intensification &   132.799000 &      3710 \\
62 &       200 &         best &          0.2 &  diversification &   202.997000 &      3529 \\
63 &       200 &         best &          0.4 &          default &   160.816000 &      3710 \\
64 &       200 &         best &          0.4 &  intensification &   692.925000 &      3703 \\
65 &       200 &         best &          0.4 &  diversification &   444.594000 &      3782 \\
66 &       200 &        first &          0.2 &          default &     0.378000 &      3234 \\
67 &       200 &        first &          0.2 &  intensification &    15.587000 &      3497 \\
68 &       200 &        first &          0.2 &  diversification &    82.407000 &      3517 \\
69 &       200 &        first &          0.4 &          default &     1.315000 &      3183 \\
70 &       200 &        first &          0.4 &  intensification &    15.661000 &      3372 \\
71 &       200 &        first &          0.4 &  diversification &    79.993000 &      3234 \\
\bottomrule
\end{tabular}
\caption{Solução obtida para cada configuração e instância do problema - parte 5.}
\label{table:all-data-5}
\end{table}

\begin{table}
\centering
\begin{tabular}{lllllll}
\toprule
{} & instances & local search & tenure ratio & method variation & running time & best cost \\
\midrule
72 &       400 &         best &          0.2 &          default &  1007.581000 &     10547 \\
73 &       400 &         best &          0.2 &  intensification &   927.970000 &     10816 \\
74 &       400 &         best &          0.2 &  diversification &   974.393000 &     10760 \\
75 &       400 &         best &          0.4 &          default &  1060.901000 &     10760 \\
76 &       400 &         best &          0.4 &  intensification &   965.190000 &     10645 \\
77 &       400 &         best &          0.4 &  diversification &  1116.701000 &     10758 \\
78 &       400 &        first &          0.2 &          default &     1.556000 &      9391 \\
79 &       400 &        first &          0.2 &  intensification &     3.751000 &      9550 \\
80 &       400 &        first &          0.2 &  diversification &    14.326000 &      9222 \\
81 &       400 &        first &          0.4 &          default &     3.369000 &      9237 \\
82 &       400 &        first &          0.4 &  intensification &     6.118000 &      9211 \\
83 &       400 &        first &          0.4 &  diversification &    14.747000 &     10254 \\
\bottomrule
\end{tabular}
\caption{Solução obtida para cada configuração e instância do problema - parte 6.}
\label{table:all-data-6}
\end{table}

\end{landscape}
 % A tabela com os resultados tá aqui


\section{Análise}
De acordo com os números apresentados, o cenário que mais se aproximou dos intervalos nos
quais os valores das soluções ótimas eram esperados foi a variação \textit{Intensification} com o \textit{best-improving}.

É notável a diferença em tempo por iteração da estratégia \textit{first-improving} em comparação com a \textit{best-improving} para as maiores instâncias. O que é esperado devido o \textit{first improving} fazer escolhas mais rápidas na média, em compensação as variações ficaram mais próximas do intervalo esperado com o \textit{best-improving}.

A influência do parâmetro \textit{Tenure} pode ser observada, onde as instâncias com 0.4 obtiveram melhor resultado do que instâncias com 0.2, na maioria dos casos, levando em consideração a busca local. Tal resultado é interessante pois mostra quanto o algoritmo deve ser restringida no problema em mãos para que o processo de busca saia de mínimos locais e explore bem o espaço de solução.

\bibliographystyle{plain}
\bibliography{bibliography.bib}

\end{document}
