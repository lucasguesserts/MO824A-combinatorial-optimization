\documentclass{article}
\usepackage[a4paper]{geometry}
\usepackage[brazil]{babel}
\usepackage[utf8]{inputenc}
\usepackage{url}
\usepackage{hyperref}
\usepackage{graphicx}
\usepackage{amsmath}
\usepackage{amsfonts}
\usepackage{amssymb}
\usepackage{amsthm}
\usepackage{xspace}
\usepackage{lscape}
\usepackage{booktabs}

\theoremstyle{definition}
\newtheorem{defn}{Definição}
\newtheorem{fact}{Fato}

\newcommand{\Set}[1]{\left\{#1\right\}}
\newcommand{\Sum}[4]{\displaystyle\sum\limits_{#1 = #2}^{#3} #4}
\newcommand{\function}[3]{#1: #2 \rightarrow #3}
\newcommand{\transpose}[1]{#1^{T}}

\newcommand{\B}{\mathbb{B}}
\newcommand{\Bn}{\mathbb{B}^n}
\newcommand{\Z}{\mathbb{Z}}
\newcommand{\Zn}{\mathbb{Z}^n}

\newcommand{\qbf}{QBF\xspace}
\newcommand{\maxqbf}{MAX-QBF\xspace}
\newcommand{\maxkqbf}{MAX-KQBF\xspace}

\newcommand{\w}{w}
\newcommand{\W}{W}

\newcommand{\aij}{a_{i,j}}
\newcommand{\A}{A}

\title{Atividade 5 - Tabu Search Aplicado ao Problema KQBF}
\author{
    Ítalo Fernandes Gonçalves RA 234990 \\
    Luiz Gustavo Silva Aguiar RA 240499 \\
    Lucas Guesser Targino da Silva RA 203534
}

\begin{document}

\maketitle

\section{Definições}

\begin{defn}[Conjunto Binário]
    $\B = \Set{0, 1}$
\end{defn}

\begin{defn}[Função Binária Quadrática (\qbf)]
    É uma função $\function{f}{\Bn}{\Z}$ da forma:
    \begin{equation}
        \label{eq:qbf}
        f(x)
        = \Sum{j}{1}{n}{x_i \cdot \aij \cdot x_j}
        = \transpose{x} \cdot \A \cdot x
    \end{equation}
    em que $\aij \in \Z, \ \forall i,j \in \Set{1, \cdots, n}$ e $\A$ é a matriz $n$ por $n$ induzida pelos $\aij$.
\end{defn}

\begin{defn}[Problema de Maximização de uma Função Binária Quadrática (\maxqbf)]
Dada uma \qbf $f$, um \maxqbf é um problema da forma:
\begin{equation}
    \label{eq:max-qbf}
    \max\limits_{x} f(x)
\end{equation}
\end{defn}

\begin{fact}
\maxqbf é NP-difícil \cite{bib:qbf}
\end{fact}

\begin{defn}[Maximum knapsack quadractic binary function (\maxkqbf)]
Dada uma \qbf $f$, um vetor $\w \in \Zn$\footnote{O problema original foi definido com números reais. Decidimos aqui utilizar inteiros por dois motivos. Primeiro, todas as instâncias fornecidas possuem apenas valores inteiros para $\aij, \w, \W$. Garante-se que os valores são sempre inteiros pois $\Z$ é fechado nas operações envolvidas: adição e multiplicação. Segundo, simplifica a implementação e comparações (não é necessário fazer comparação de números em ponto flutuante).}, e um valor $\W \in \Z$, um \maxkqbf é um problema da forma:
\begin{eqnarray*}
    \label{eq:max-kqbf}
    \max & f(x) \\
    \mbox{subjected to} & \transpose{\w} x \leq \W \\
    & x \in \Bn
\end{eqnarray*}
\end{defn}


\begin{landscape}

\begin{table}
\centering
\begin{tabular}{lllllll}
\toprule
{} & instances & local search & tenure ratio & method variation & running time & best cost \\
\midrule
0  &       020 &         best &          0.2 &          default &     1.263000 &        93 \\
1  &       020 &         best &          0.2 &  intensification &     0.447000 &        93 \\
2  &       020 &         best &          0.2 &  diversification &     0.246000 &        93 \\
3  &       020 &         best &          0.4 &          default &     0.997000 &       120 \\
4  &       020 &         best &          0.4 &  intensification &     0.476000 &       120 \\
5  &       020 &         best &          0.4 &  diversification &     0.255000 &       120 \\
6  &       020 &        first &          0.2 &          default &     0.043000 &        93 \\
7  &       020 &        first &          0.2 &  intensification &     0.035000 &       104 \\
8  &       020 &        first &          0.2 &  diversification &     0.031000 &        93 \\
9  &       020 &        first &          0.4 &          default &     0.057000 &       102 \\
10 &       020 &        first &          0.4 &  intensification &     0.205000 &       110 \\
11 &       020 &        first &          0.4 &  diversification &     0.035000 &       104 \\
\bottomrule
\end{tabular}
\caption{Solução obtida para cada configuração e instância do problema - parte 0.}
\label{table:all-data-0}
\end{table}

\begin{table}
\centering
\begin{tabular}{lllllll}
\toprule
{} & instances & local search & tenure ratio & method variation & running time & best cost \\
\midrule
12 &       040 &         best &          0.2 &          default &     2.660000 &       239 \\
13 &       040 &         best &          0.2 &  intensification &     1.633000 &       290 \\
14 &       040 &         best &          0.2 &  diversification &     1.284000 &       260 \\
15 &       040 &         best &          0.4 &          default &     2.707000 &       308 \\
16 &       040 &         best &          0.4 &  intensification &     4.774000 &       316 \\
17 &       040 &         best &          0.4 &  diversification &     1.218000 &       303 \\
18 &       040 &        first &          0.2 &          default &     0.049000 &       201 \\
19 &       040 &        first &          0.2 &  intensification &     0.090000 &       239 \\
20 &       040 &        first &          0.2 &  diversification &     0.195000 &       243 \\
21 &       040 &        first &          0.4 &          default &     0.114000 &       239 \\
22 &       040 &        first &          0.4 &  intensification &     0.147000 &       239 \\
23 &       040 &        first &          0.4 &  diversification &     0.224000 &       239 \\
\bottomrule
\end{tabular}
\caption{Solução obtida para cada configuração e instância do problema - parte 1.}
\label{table:all-data-1}
\end{table}

\begin{table}
\centering
\begin{tabular}{lllllll}
\toprule
{} & instances & local search & tenure ratio & method variation & running time & best cost \\
\midrule
24 &       060 &         best &          0.2 &          default &     5.390000 &       368 \\
25 &       060 &         best &          0.2 &  intensification &     4.510000 &       480 \\
26 &       060 &         best &          0.2 &  diversification &     3.904000 &       483 \\
27 &       060 &         best &          0.4 &          default &     5.427000 &       491 \\
28 &       060 &         best &          0.4 &  intensification &     5.166000 &       446 \\
29 &       060 &         best &          0.4 &  diversification &     5.662000 &       491 \\
30 &       060 &        first &          0.2 &          default &     0.066000 &       413 \\
31 &       060 &        first &          0.2 &  intensification &     0.238000 &       452 \\
32 &       060 &        first &          0.2 &  diversification &     0.715000 &       406 \\
33 &       060 &        first &          0.4 &          default &     0.163000 &       408 \\
34 &       060 &        first &          0.4 &  intensification &     0.327000 &       397 \\
35 &       060 &        first &          0.4 &  diversification &     3.064000 &       455 \\
\bottomrule
\end{tabular}
\caption{Solução obtida para cada configuração e instância do problema - parte 2.}
\label{table:all-data-2}
\end{table}

\begin{table}
\centering
\begin{tabular}{lllllll}
\toprule
{} & instances & local search & tenure ratio & method variation & running time & best cost \\
\midrule
36 &       080 &         best &          0.2 &          default &    10.182000 &       683 \\
37 &       080 &         best &          0.2 &  intensification &     7.984000 &       662 \\
38 &       080 &         best &          0.2 &  diversification &     7.651000 &       780 \\
39 &       080 &         best &          0.4 &          default &    10.318000 &       783 \\
40 &       080 &         best &          0.4 &  intensification &     8.769000 &       732 \\
41 &       080 &         best &          0.4 &  diversification &     6.590000 &       702 \\
42 &       080 &        first &          0.2 &          default &     0.089000 &       667 \\
43 &       080 &        first &          0.2 &  intensification &     0.513000 &       674 \\
44 &       080 &        first &          0.2 &  diversification &     2.313000 &       680 \\
45 &       080 &        first &          0.4 &          default &     0.273000 &       692 \\
46 &       080 &        first &          0.4 &  intensification &     0.649000 &       692 \\
47 &       080 &        first &          0.4 &  diversification &     2.205000 &       692 \\
\bottomrule
\end{tabular}
\caption{Solução obtida para cada configuração e instância do problema - parte 3.}
\label{table:all-data-3}
\end{table}

\begin{table}
\centering
\begin{tabular}{lllllll}
\toprule
{} & instances & local search & tenure ratio & method variation & running time & best cost \\
\midrule
48 &       100 &         best &          0.2 &          default &    17.558000 &      1220 \\
49 &       100 &         best &          0.2 &  intensification &    15.455000 &      1116 \\
50 &       100 &         best &          0.2 &  diversification &   130.585000 &      1193 \\
51 &       100 &         best &          0.4 &          default &    17.549000 &      1227 \\
52 &       100 &         best &          0.4 &  intensification &    28.078000 &      1249 \\
53 &       100 &         best &          0.4 &  diversification &    17.294000 &      1225 \\
54 &       100 &        first &          0.2 &          default &     0.118000 &       979 \\
55 &       100 &        first &          0.2 &  intensification &     1.041000 &       966 \\
56 &       100 &        first &          0.2 &  diversification &     5.663000 &       850 \\
57 &       100 &        first &          0.4 &          default &     0.350000 &      1121 \\
58 &       100 &        first &          0.4 &  intensification &     1.519000 &      1128 \\
59 &       100 &        first &          0.4 &  diversification &     6.309000 &       878 \\
\bottomrule
\end{tabular}
\caption{Solução obtida para cada configuração e instância do problema - parte 4.}
\label{table:all-data-4}
\end{table}

\begin{table}
\centering
\begin{tabular}{lllllll}
\toprule
{} & instances & local search & tenure ratio & method variation & running time & best cost \\
\midrule
60 &       200 &         best &          0.2 &          default &   123.439000 &      3947 \\
61 &       200 &         best &          0.2 &  intensification &   132.799000 &      3710 \\
62 &       200 &         best &          0.2 &  diversification &   202.997000 &      3529 \\
63 &       200 &         best &          0.4 &          default &   160.816000 &      3710 \\
64 &       200 &         best &          0.4 &  intensification &   692.925000 &      3703 \\
65 &       200 &         best &          0.4 &  diversification &   444.594000 &      3782 \\
66 &       200 &        first &          0.2 &          default &     0.378000 &      3234 \\
67 &       200 &        first &          0.2 &  intensification &    15.587000 &      3497 \\
68 &       200 &        first &          0.2 &  diversification &    82.407000 &      3517 \\
69 &       200 &        first &          0.4 &          default &     1.315000 &      3183 \\
70 &       200 &        first &          0.4 &  intensification &    15.661000 &      3372 \\
71 &       200 &        first &          0.4 &  diversification &    79.993000 &      3234 \\
\bottomrule
\end{tabular}
\caption{Solução obtida para cada configuração e instância do problema - parte 5.}
\label{table:all-data-5}
\end{table}

\begin{table}
\centering
\begin{tabular}{lllllll}
\toprule
{} & instances & local search & tenure ratio & method variation & running time & best cost \\
\midrule
72 &       400 &         best &          0.2 &          default &  1007.581000 &     10547 \\
73 &       400 &         best &          0.2 &  intensification &   927.970000 &     10816 \\
74 &       400 &         best &          0.2 &  diversification &   974.393000 &     10760 \\
75 &       400 &         best &          0.4 &          default &  1060.901000 &     10760 \\
76 &       400 &         best &          0.4 &  intensification &   965.190000 &     10645 \\
77 &       400 &         best &          0.4 &  diversification &  1116.701000 &     10758 \\
78 &       400 &        first &          0.2 &          default &     1.556000 &      9391 \\
79 &       400 &        first &          0.2 &  intensification &     3.751000 &      9550 \\
80 &       400 &        first &          0.2 &  diversification &    14.326000 &      9222 \\
81 &       400 &        first &          0.4 &          default &     3.369000 &      9237 \\
82 &       400 &        first &          0.4 &  intensification &     6.118000 &      9211 \\
83 &       400 &        first &          0.4 &  diversification &    14.747000 &     10254 \\
\bottomrule
\end{tabular}
\caption{Solução obtida para cada configuração e instância do problema - parte 6.}
\label{table:all-data-6}
\end{table}

\end{landscape}


\bibliographystyle{ieeetr}
\bibliography{bibliography}

\end{document}
