\documentclass{article}
\usepackage[a4paper]{geometry}
\usepackage[utf8]{inputenc}
\usepackage{url}
\usepackage{hyperref}

\title{Atividade 1}
\author{Lucas Guesser Targino da Silva}
\date{\today}

\newcommand{\Set}[1]{$\left\{#1\right\}$}
\newcommand{\Sum}[1]{\displaystyle\sum\limits_{#1}}

\begin{document}

\maketitle

\section{Modelo Matemático}

\subsection{Variáveis de Decisão}

\begin{itemize}
	\item $x_{p,l,f}$: quantidade, em toneladas, do produto $p$ produzida na máquina $l$ da fábrica $f$.
	\item $y_{p,f,j}$: quantidade, em toneladas, do produto $p$ transportada da fábrica $f$ para o cliente $j$.
\end{itemize}


\subsection{Objetivo}

Minimizar:
\begin{equation}
 	\Sum{p}\Sum{l}\Sum{f} p_{p,l,f} \  x_{p,l,f}
	+
	\Sum{p}\Sum{f}\Sum{j} t_{p,l,f} \  y_{p,l,f}
\end{equation}

Sujeito a:
\begin{equation}
	\label{constraint:demand}
	D_{j,p} \leq \Sum{f} y_{p,f,j}
	\quad \forall p \ \forall j
\end{equation}

\begin{equation}
	\label{constraint:resources}
	R_{m,f} \geq \Sum{p} \Sum{l} r_{m,f,l} \ x_{p,l,f}
	\quad \forall f \ \forall m
\end{equation}

\begin{equation}
	\label{constraint:production-capacity}
	C_{l,f} \geq \Sum{p} x_{p,l,f}
	\quad \forall l \ \forall f
\end{equation}

\begin{equation}
	\label{constraint:compatibility}
	\Sum{l} x_{p,l,f} = \Sum{j} y_{p,l,f}
	\quad \forall p \ \forall f
\end{equation}

A notação $\forall i$ acima (podendo $i$ ser $p, l, f, m, j$) significa que a restrição se aplica a todos os valores do domínio discreto de $i$. Por exemplo, se o domínio de $f$ for \Set{1, 2, 3}, então a condição se aplica para $f=1$, $f=2$ e $f=3$.


Esse modelo ainda não usa \cite{bib:gurobi}

\section{Referências}

\bibliographystyle{unsrt}
\bibliography{bibliography}

\end{document}
